
\documentclass[a4paper,12pt,twoside]{article}
\usepackage[american]{babel}
\usepackage{rotating}
\usepackage{epsfig}
\usepackage[centertags]{amsmath}
%\usepackage{amssymb,amscd,./styles/amsatdef}
\usepackage{amscd}
\usepackage{array}
\usepackage{multirow}
\usepackage{supertabular}
\usepackage{colordvi}
%axodraw}
\usepackage{dcolumn}
%\usepackage{./styles/mcite}
%\usepackage{./styles/citesort}
\usepackage{xspace}
\usepackage{lscape}
%
%       File 'zeuspap_def.tex' contains all TeX definitions etc.
%
%\input zeus_def/zeuspap_def.tex
\newcommand{\ptmiss}{{\mbox{$\not\hspace{-.55ex}{P}_T$}}}
%%\newcommand{\pbi}{{$pb^{-1}$}}
%\newcommand{\nubar}{{\bar{\nu}}}
%
%       File 'zeuspap_fmt.tex' contains all format parameters
%       like page layout, line spacing etc.
%
%\input zeus_def/zeuspap_fmt.tex
%
%       Now set draft date...
%
\def\drftdate{{\today}}
%==============================================================================
%       Here comes the document.
%==============================================================================
\begin{document}
\selectlanguage{american}
%------------------------------------------------------------------------------
%       Title page
%------------------------------------------------------------------------------
\thispagestyle{empty}
%--- Here come the title lines indicating conference, abstract number, sessions
\title{
%\zeusconfpaphead%
%--- Name of the conference:
{}
%--- Place of the conference:
{}
%--- Abstract number:
{}
%--- Plenary session line:
{}
%--- Parallel session line:
{}
\vskip2.2cm
%--- Here comes the paper title
{\bf MaGe Evaluation Report }
}

                    
\author{Contact: Xiang Liu, Luciano Pandola \\
                 Version 0.5}
%--- Use \date{\drftdate} if you want to indicate the date on the title page
\date{Version 0.0}
\date{\drftdate}
\maketitle

\vfill
\centerline{\bf Abstract}
\vskip4.mm
\centerline{
  \begin{minipage}{15.cm}
    \noindent
In this note we report about our evaluation of the Majorana Monte Carlo 
package called MJ. In particular, we focused on  
its suitability for the future joint use and development by the 
Majorana and Gerda MC groups.
We received a released version of the MJ package 
by the courtesy of the Majorana MC group. We then made some 
modifications and implemented a simplified Gerda geometry.
The updated simulation package for both Collaborations is named ``MaGe''.
During the work we find that the existing package is well-designed, 
easy to use and flexible for modifications.\\
Therefore, we propose that the Gerda Task Group T10 
(\emph{Simulation and background studies}) should 
collaborate with the Majorana group for the development of a 
common Geant4-based MC framework.  
\end{minipage}
  }
\vfill

\thispagestyle{empty}
%\newpage


\clearpage
\tableofcontents
\clearpage

\clearpage
\newpage
\section{Introduction}

After the Majorana-Gerda MC workshop at Gran Sasso,
we were assigned the task to evaluate the Majorana package.
With the help of the Majorana group, we received the source 
code of the existing MJ package and developed the first 
version of MaGe, a common MC package for both Collaborations. 
We are convinced that MaGe is the most viable choice for 
the Gerda MC group.

The note is arranged in the following parts.
The recent Gerda MC activities are summarized in Chapter 
\ref{sec:gerda-mc-ongoing-activity}.
Chapter~\ref{sec:gerda-mc-needs-a-common-framework} then 
describes why a common and well-designed framework is needed  
by the Gerda MC developers for a more efficient work and collaboration. 
Chapter~\ref{sec:why-majorana-mc-package} summarizes
the main reasons of our proposal to adopt, extend and develop the 
existing Majorana framework.
Details about the framework are given in 
Chapter~\ref{sec:what-is-in-majorana-mc-package}.
Chapter~\ref{sec:first-version-of-mage}
summarizes the work and the contributions made so far in the 
MaGe package. 
Chapter~\ref{sec:cvs-and-database-servers} lists
our suggestions concerning CVS repository and database issues.
Detailed instructions of how to implement a new geometry
can be found in the appendix.


%\clearpage
%\newpage
\section{Gerda MC ongoing activity}
\label{sec:gerda-mc-ongoing-activity}

A short summary of the ongoing MC activities in the different  
Institutions of the Gerda Collaboration is presented below.
They are extracted so far from the presentations given at the LNGS
MC workshop, so are by no means complete.

\begin{itemize}

\item Luciano Pandola (LNGS)
General framework designs.

\item Xiang Liu (MPI-Munich)
Radiopurity requirements for the Gerda
support structures and front end electronics.
Suppression of the internal
and surface background of the Ge detectors.
Future simulation of the test facilities that will be built 
in Munich.

\item Michael Bauer (T\"{u}bingen)
Cosmic-ray-induced neutron and isotope background from the surrounding
environment. A comparison
between Geant4 and FLUKA was already accomplished.

\item Stephan Scholl (T\"{u}bingen)
Low-energy neutron recoil with Geant4; feasibility of dark matter 
search with Gerda.
%To study low energy neutron recoil simulations in Geant4, and
%to study the feasibility of using Gerda setup for
%dark matter search. 
%Through this study, we could have 
%a better understanding of
%of the background events induced by the inelastic scattering from
%ambient neutron flux.

\item Hardy Simgen (Heidelberg)
$^{39}$Ar background in liquid argon; feasibility study
for dark matter search.


\item George Rugel (Heidelberg)
Simulation of the detector GeMPI-II; the software is presently independent 
on Geant4. GeMPI-II will be used for the Gerda material screening. 
If a new Geant4-base simulation is found to be necessary, 
MaGe can accommodate it.


\end{itemize}

%\clearpage
%\newpage
\section{Gerda MC needs a common framework }
\label{sec:gerda-mc-needs-a-common-framework}

\subsection{A common framework}

Geant4\cite{G4ref} is a toolkit for the Monte Carlo simulation of the interaction 
of particles with matter, based on the Object-Oriented technology. 
It allows to divide 
the simulation chain into de-coupled classes: material, 
geometry, event generator, physics processes, output and 
visualization.\\
%The names of these classes are self-explainatory. 
Users are required to write their own geometry input,
select the physics processes to be activated and define 
their own output format.
Once these are done, the Geant4 interfaces can manage the 
simulation process and the communications among the different 
components. \\

Gerda MC Task Group needs a common framework able to
\begin{enumerate}
\item provide the complete simulation chain;
\item facilitate the cooperation among the TG members and 
allow the distributed development of the software.
\item make comparisons between different studies possible,
as the same basic tools are used.
\end{enumerate} 

A flexible framework providing a suitable set of 
interfaces and base classes is hence required. In this case the 
development can proceed in a modular way: each MC 
developer implements only the specific and concrete classes required for 
his/her study. The framework ensures the correct 
``communication'' among several de-coupled modules of the software, 
so that the interfaces do no need to be re-written from scratch.
In this way the developers can effectively avoid the duplication of work
and cooperate efficiently.

\subsection{Gerda subgroups in common framework}

When the common framework for Gerda is ready and the 
default base classes are defined, the 
Institutions can develop different specific modules 
in parallel 
and contribute to the complete simulation chain without 
overlaps. This is summarized in Table~\ref{needs_table}.

It will be shown later that the MaGe 
represents a suitable framework meeting all the 
requirements of flexibility and modularization 
described above. 

\subsection{Gerda MC future needs}
The complete simulation of pulse shapes will also be needed in the
future. The requirement can be met in two ways:
\begin{itemize}
\item with a native (or re-engineered) software integrated in the MaGe
framework and directly interfaced with the MC tracking;
\item with a stand-alone software (like the FORTRAN-based one
developed by the Padua group) which takes as input the output
file (energy deposit, etc.) of the MC stage.
\end{itemize}
%Then the full event information, including energy deposit as function
%of time, mirror charge etc, can be used for further optimizing
%detector arrangements and segments.

For the long term, once Gerda starts operating, many more things are needed.
\begin{itemize}
\item Online trigger and DAQ simulation.
\item Database for calibrations of individual Ge crystals,
including energy scale, geometry, HV, alignment, properties
for mirror charge and pulse shape.
\item The format of Output data of each triggered event to be saved.
\item Off-line analysis tools and access to database.
\end{itemize}

All these items should be included in the Gerda software package
so that online and off-line analysis are consistent, and
data and MC events can also be compared consistently. The issue
has to be taken jointly by the T9 (\emph{DAQ electronics and
software}) and T10 Gerda working groups.


\begin{landscape}
\begin{table}[tb]
\begin{center}
%%%\begin{tabular}{|c||c|c|c|c|c|}
\begin{tabular}{|m{2.5cm}||m{3.2cm}|m{3.2cm}|m{3.2cm}|m{3.2cm}|m{3.2cm}|}
\hline
     & Material & Geometry & Event generator & Physics process
     & Output   \\ \hline\hline
Default             & definition of normal materials and their components
                    & whole Gerda setup, 
               including crystal , cryogenics, supporting and shielding.
                    & Geant4 particle generations with most
                      radioactive isotopes and their decay chain
                    & Geant4 simulation of particles interacting
                      in detector and shielding materials
                    & Root ntuples with energy deposit information \\ \hline
LNGS                & default & default
                    & interface for 0$\nu$- and 2$\nu$- 2$\beta$ 
                      decays from Decay0 package $\star$
                    & default & generalize to other analysis tools than ROOT $\star$ \\ 
\hline
Background Munich   & materials for supporting structure
                    & provide default Gerda setup $\star$
                    & default & default 
                    & provide trajectories and points $\star$ \\ \hline
Test-facility Munich & default
                    & own geometry
                    & default & default & own output \\ \hline
Neutron T\"{u}bingen& default      & default
                    & provide neutron flux 
                    & study neutron interaction
                    & default         \\ \hline
$^{39}$Ar Heidelberg     & special liquid argon if necessary
                    & liquid Ar cryogenic structure
                    & default      & optical photon tracking
                    & default         \\ \hline
GeMPI-II Heidelberg & new material definition if necessary &  own geometry
                    & default      & default
                    & own output      \\ \hline
To be defined                 & default      & default
                    & default      & mirror charge and pulse shape simulation
                    & energy deposit as function of time   \\ \hline
\end{tabular}
\caption{Foreseen contributions from the different Institutions to the common
framework. $\star$ for already accomplished.} \label{needs_table}
\end{center}
\end{table}
\end{landscape}

%\clearpage
%\newpage
\section{Why Majorana MC package}
\label{sec:why-majorana-mc-package}
We suggest to collaborate with Majorana MC group
and share one single MC package, MaGe. Our suggestion
is based on the following considerations.

%---------------------------------------------------------------%
%Luciano: Here I also try to stress what Majo gains from the collaboration
%Xiang:   I combined the first and third points as they are essentially same
%Luciano: Ok, it sounds good.
%---------------------------------------------------------------%

\begin{itemize}
\item Gerda needs a common framework and Majorana provides one
with good organization and enough flexibility. The collaboration   
is of reciprocal advantage, because Gerda and Majorana
are studying the same physics process. The development of common 
tools (e.g. generators, pulse shape simulation) is shared and 
not-duplicated, while each side takes care independently 
of the specific concrete parts (e.g. geometry). 

\item The present MJ interfaces are flexible enough that they does not 
put \emph{any} constraint to the Gerda side concerning geometry, physics, 
i/o and generators. All the detector-specific or 
not-suitable existing components can be extended or replaced with others 
written from scratch. Gerda is hence not forced to use any of the 
existing components: each of them can be independently re-written, 
sharing only the generic interfaces. 
%--------------------------------------------------------------------%
% Luciano:
% I tried to stress this point. I don't know if it is clear and/or if 
% is too underlined.
% Xiang: I agree with your point.
%--------------------------------------------------------------------%

\item More data from
real experiments will be available for the MC validation. 
Since both Gerda and Majorana have their own R\&D research
and their own measurements from different test facilities,
simulations can be compared and studied more extensively.

\item Geant4 is flexible enough for possible framework modifications
and is suitable for distributed development.

\end{itemize}

%\clearpage
%\newpage
\section{What is in Majorana MC package?}
\label{sec:what-is-in-majorana-mc-package}

\subsection{Framework provided}
The most important thing about the Majorana MJ package
is that the framework is already designed, implemented 
and working.
After getting the source code, we could define our own geometry 
and output format and start the events simulation. 
A key point is that only one executable 
file is created and the user can choose at run-time (via macros) 
the specific implementation he/she wants for geometry, physics, 
generators and i/o.
As an example, the procedure for the definition of the 
default Gerda geometry and a steering file for loading this geometry
are described in the next chapter.

The following directories are present in the MJ package:
\begin{itemize}
\item {\tt geometry}
\item {\tt materials}
\item {\tt database}
\item {\tt generators}
\item {\tt processes}
\item {\tt io}
\item {\tt management}
\item {\tt waveform}
\end{itemize}

They are described in the following sections.
In each section, possible future developments
are also briefly discussed.

\subsection{Geometry ({\tt geometry}) }
It contains a virtual base geometry class for the interface with 
the rest of the framework.
Concrete classes for new geometries just need to inherit the base 
class and define the specific volumes.

Majorana implemented two experimental setups, describing a clover 
detector and a clover detector inside a NaI barrel calorimeter. 

\subsection{Material ({\tt material}) }
Majorana provides a small material database of 22 items, based 
on postgreSQL. All materials defined in the database are automatically 
constructed in the simulation; no additional code is required to 
add new materials. It is hence easy to define and use materia;s.

\subsection{Event generator ({\tt generators})}
It contains a virtual base generator class that manages the general interface 
with the framework. New customized generators inherit this base class. 
Presently 5 alternative generators are available. 
Our first concrete contribution to the common package was the implementation of  
the interface for the DECAY0 generator, which is working and available for 
Majorana collaborators.
Majorana uses the PNNL database for radioactive isotopes and decays.
Have to be checked whether it is possible to it.
We foresee to implement a new generator for the simulation of cosmic-ray 
muons inside the Gran Sasso Lab and a new algorithm to randomly sample 
a point within a volume (a preliminary version is already available).

\subsection{Physics process ({\tt processes})}
Majorana defines a unique physics list with two different sets of 
production cuts: one for dark matter search (low energy threshold), 
the other for standard double-beta decay search 
(higher energy threshold and faster speed). Initial 
studies are done showing
that higher energy threshold does not affect physics results
for double-beta decay. 
We would like to have the possibility to switch among 
alternative physics lists by macro commands, in order to optimize the 
hadronic processes. This option is still not available but could be implemented 
in a straightforward way. A further possible development line is the definition of 
different cuts for the more-interesting (e.g. germanium detectors) and 
the less-interesting (e.g. air outside the shielding) regions of the set-up, in 
order to optimize the CPU-time. 
%-----------------------------------------------------------%
%Luciano: Do you think it should put here?
% Xiang:  I agree.
%-----------------------------------------------------------%

\subsection{Output ({\tt io}) }

The event informations need to be saved on ntuples, or ASCII 
files or whatever else. The base class for the actions
at run level, event level and step level is written.
User just needs to define the set of variables and
the ways to retrieve and calculate them at each level. 
The module was extended in order to accomodate other 
analysis tools alternative to ROOT: the top-level 
base class for the i/o management is no longer ROOT-specific 
(as it was in the original MJ package) but it is now  
completely generic.

\subsection{Macro control ({\tt management})}
The MJ package generates only one executable;  
macro commands can be used to set geometry, tune geometry parameters,
select physics process, choose event generator and output scheme. 
It is fairly easy to add more macro commands if needed.
Furthermore, the root of the package class hierarchy ({\tt MJManager}) 
is a singleton: any class can be accessed via the 
manager, avoiding the need for global variables. 

\subsection{Preliminary waveform simulation ({\tt waveform})}
This module (not ready yet) is called pulse shape simulation
within Gerda collaboration 
and is of general interest and the 
development could then be shared between Majorana and Gerda, 
with mutual benefit.
Standalone waveform simulation codes are available at the
semiconductor Lab in Munich and at the INFN-Padua. They will be 
tested in the future and possible interfaced with the MC side.
The Majorana group is independently working on pulse shape 
simulation as well.
%-------------------------------------------%
%Luciano: Should we take this subsection?
%Xiang: maybe we could mention Padova and Munich here
%Luciano: Added something. Check if it sounds good, please.
%-------------------------------------------%


\subsection{How to implement a new geometry}
The following is a quick guide for the implementaytion of new geometries
into the existing Majorana package; it shows that the procedure is
easy to understand and to work with. 
More details about the Gerda geometry already implemented 
in MaGe can be found in the appendix.

Two steps are required.

First: the new geometry needs to be added.
\begin{itemize}
\item add the definitions for the needed elements and materials in
the material table, i.e. in the database or in the local 
file {\tt MaGeGerdaLocalMaterialTable} under {\tt materials}
\item Write your own geometry class, which returns the pointer of
your main (i.e. outermost) logical volume. The world volume 
is a 10$\times$10$\times$10 m cube filled of air and it is 
created by default. The logical volume provided by the user has to 
contain the whole set-up is placed at the center of the 
world box.
\item Write your own sensitive detector classes and associate them
to your geometry.
\item Register the geometry in the main framework
(by modifying one line and adding two new lines of code)
so that
it can be loaded at run-time via macro command.
\end{itemize}

Second: save the event information to ntuples
after new geometry is added.
\begin{itemize}
\item Write your i/o class which is inherited from class 
{\tt MJOutputROOT }
(if ROOT-based output is wanted, otherwise from {\tt MJVOutputManager} ).
Within this class override the following member functions:

\begin{itemize}
\item {\tt BeginOfRunAction()}
(opens file and defines variables to be saved to ROOT ntuples.)
\item {\tt BeginOfEventAction()}
(retrieves true MC informations for each event.)
\item {\tt EndOfEventAction()} 
(calculates variables after the event is finished.)
\item {\tt EndOfRunAction()} (closes file.)
\item {\tt RootSteppingAction()}
(retrieves and saves information after each Geant4 stepping; it can
be left empty.)
\end{itemize}

\item Register your i/o class in the main framework (by modifying one line
and adding 4 more lines of code) so that the output scheme can
be selected by macro commands.
\end{itemize}

After these, the new geometry can be loaded at run time and
results will be saved to ntuples.

%\clearpage
%\newpage

\section{First version of MaGe}
\label{sec:first-version-of-mage}

A test version of MaGe with the Gerda geometry is available.

\subsection{List of work already done in MaGe}
The preliminary MaGe package is based on the original MJ simulation package
with the following modifications.
\begin{itemize}
\item Implementation of Gerda-specific geometry and output inside two
new directories {\tt gerdageometry} and {\tt gerdaio} (see next sections
 for more details).
\item Accessing trajectories of all particles.
\item Interface to the 2$\nu$-2$\beta$ and 0$\nu$-2$\beta$ event generator DECAY0.
\item Generation of events with random distribution inside some volumes.
(preliminary)
\item Construction of a local material database and registration in 
the main frame.
\item Generalization of the output interface to beyond ROOT.
\end{itemize}

\subsection{Gerda geometry in MaGe}

The implemented Gerda geometry contains 21 Ge crystals surrounded 
by different shielding materials, as shown in Fig.~\ref{fig:gerdasetup}.
\begin{figure}[h]
\centerline{  \epsfxsize=2.0in  \epsfbox{gerda_picture/crystal.eps}
              \epsfxsize=3.5in  \epsfbox{gerda_picture/watertank.eps}}
\vskip 0 cm
\caption{Left plot shows the size of one true coaxial Ge crystal,
right plot shows the Gerda setup with liquid N$_{2}$ shielding inside and
water shielding outside. (not to scale)}
\label{fig:gerdasetup}
\end{figure}
The single Ge crystal is true coaxial with the size shown 
in Fig.~\ref{fig:gerdasetup}.
The Ge array has 3 layers with 7 Ge crystals in each layer.
The vertical gap between two layers is 5 cm, and the horizontal
gap between two neighboring crystals within the same layer is 1 cm.


\subsection{One steering file}

The following macro steering file is an example of generating
0$\nu$-2$\beta$ events with random distributions
inside one Ge crystal. \\
1 {\tt/MJ/manager/mjlog trace} \\
2 {\tt/MJ/geometry/detector GerdaArray} \\
3 {\tt/MJ/geometry/database false} \\
4 {\tt/MJ/processes/realm BBdecay} \\
5 {\tt/MJ/eventaction/rootschema GerdaArrayWithTrajectory} \\
6 {\tt/MJ/eventaction/rootfilename testGerda.root} \\
7 {\tt/MJ/eventaction/reportingfrequency 1000} \\
8 {\tt/run/initialize} \\
9 {\tt/MJ/generator/confine volume} \\
10 {\tt/MJ/generator/volume Ge\_det\_0} \\
11 {\tt/MJ/generator/select decay0} \\
12 {\tt/MJ/generator/decay0/filename ge76.0nubb.1} \\
13 {\tt/tracking/verbose 0} \\
14 {\tt/run/beamOn 10000} \\

Line 1 defines the log-book mode. Line 2 selects the Gerda geometry.
Line 3 disables the connection to the LBL database
(the local database will be used). 
Line 4 selects the physics processes for 2$\beta$ decay study.
Line 5 selects the set of variables to be saved to ntuples with
the file name specified in line 6.
Line 7 asks for reporting the event information for every 1000 events.
Line 8 triggers the run initialization (material table and 
geometry are actually loaded). 
Line 9 asks the events to be generated randomly
inside some volume with the name specified in line 10.
Line 11 selects decay0 as event generator with the interface filename
specified in Line 12. Line 13 asks for no debugging information during 
the tracking. Line 14 then starts running for 10,000 events. 
%
The simulation reproduced the results from
an independent previous simulation.



%\clearpage
%\newpage
\section{CVS and database servers}
\label{sec:cvs-and-database-servers}

A common MaGe CVS repository is needed for Majorana and Gerda people to 
cooperate. Originally, we suggested two synchronized and independent 
CVS repositories, one at LBL, one at Munich. 
%The synchronization will happen 
%on a daily base. 
However, this proposal turned out to be technically unfeasible.
Therefore,
we propose to create a single main repository (e.g. at Munich) which is 
regularly backupped by the other side. 
Presently a preliminary CVS repository for Gerda collaborators is available 
at Munich and is being tested extensively by both Gerda and Majorana
people.
%-------------------------------------------%
%Luciano: It is the case to add the instruction of how to download a local 
%         copy of the repository?
%-------------------------------------------%
We also suggest to have two separate databases, one at LBL, one at Munich.
Users will be able to select which database to connect to through macro 
commands. The two database could be synchronized weekly.


Right now the MaGe code can be checked out by the command \\
{\tt setenv CVSROOT :pserver:gerdaguest@pclg-02.mppmu.mpg.de:/home/pclg-02/cvsroot \\
cvs login } (enter password {\tt gerda}) \\
{\tt cvs checkout MaGe \\
cvs logout}  \\
Then follow the instructions in \\
{\tt MaGe/doc/MaGeSetup/setup.ps} \\
to compile the code.



\section{Conclusion}
Having investigated in depth the Majorana MC package and successfully 
instantiated a preliminary Gerda geometry with the corresponding output,
we conclude that the Majorana package provides a simulation framework
which is well-designed, easy to use, flexible and suitable for future
collaborations.
Therefore, we suggest to collaborate with the Majorana MC developers 
and to construct a common software package named MaGe.
Each side will have its own geometry setup, output 
format and specific modules, sharing the development and 
the know-how of general interfaces and common tools.

\clearpage
\newpage
\appendix

\section{Implementing Gerda geometry}
This section describes in detail how the Gerda geometry, including
crystals and shielding materials, is implemented in the
first version of MaGe. It serves both as an example and as 
a manual to create new geometries in MaGe.

{\bf Step 1. Create Gerda geometry }

One new directory {\tt gerdageometry} is created with
the following new files: 
\begin{itemize} 
\item {\tt MaGeGeometryGerda} calls the following classes and
contains the whole Gerda setup. It inherites the base 
detector class {\tt MJGeometryDetector}
and overrides the member function {\tt ConstructDetector()}.
It sets the logical volume pointer of the whole detector
to the variable {\tt theDetectorLogical} defined in {\tt MJGeometryDetector}.
\item {\tt MaGeGeometryShielding} describes the shielding structure.
\item {\tt MaGeGeometryGermaniumArray} contains the array of crystals.
\item {\tt MaGeGeometryGermaniumCrystal} defines one single crystal.
\end{itemize}

The size of the detectors can be tuned as shown in the corresponding
{\tt Messenger} classes.
New elements and materials can be added in
class \\
{\tt MJGerdaLocalMaterialTable} under directory
{\tt materials}.


{\bf Step 2. Define sensitive volume}
The class {\tt MaGeGeometrySD} is a general class for sensitive
volumes and is used in class {\tt MaGeGeometryGermaniumArray}
to define the Ge crystals sensitive.
%-------------------------------------------%
%Luciano: This is for future: also water & nitrogen buffers
%         should have a sensitive volume (at least the energy 
%         deposit).
%-------------------------------------------%

{\bf Step 3. Register the Gerda geometry}
In class {\tt MJGeometryDetectorMessenger},
include name {\tt GerdaArray} for the new geometry in the line: \\
{\tt MJGeometryDetectorChoiceCommand->SetCandidates} \\
so that it can be loaded by macro.

Then in class {\tt MJGeometryDetectorConstruction},
add the following lines in the member function {\tt SetDetector()} \\
{\tt~~~~~      else if (detectorType == "GerdaArray") \\
    ~~~~~      theDetector = new MaGeGeometryGermaniumArray(); \\
}
This defines the actions for the macro command, which is to
instantiate the new geometry.

{\bf Step 4. Create Gerda output}
One new directory {\tt gerdaio} is created with the following file \\
{\tt MaGeOutputGermaniumArray} with the following member functions
\begin{itemize}
\item {\tt BeginOfRunAction()} 
\item {\tt DefineSchema()} defines the set of variables to be saved.
\item {\tt BeginOfEventAction()} 
\item {\tt EndOfEventAction()}
\item {\tt EndOfRunAction()}
\item {\tt RootSteppingAction()}
\end{itemize}
which are already explained in Chapter~\ref{sec:what-is-in-majorana-mc-package}.


{\bf Step 5. Register Gerda output}
In class {\tt MJManagementEventActionMessenger} under directory
{\tt management}, include the new output schema {\tt GerdaArray}
in the line \\
{\tt schemacandidates = ...} \\
Then in member function {\tt SetNewValue()}, add the following lines \\
{\tt
    else if (newValues == "GerdaArray") \{ \\
      fEventAction->SetOutputManager(new MaGeOutputGermaniumArray()); \\
      fEventAction->SetOutputName("GerdaArray"); \} \\
}
Notice that {\tt GerdaArrayWithTrajectory} is a different schema
requiring the trajectory information to be saved to ntuple as well.

{\bf Step 6. Add new commands in macro }

Add the following new commands in the macro \\
{\tt  
/MJ/geometry/detector GerdaArray \\
/MJ/eventaction/rootschema GerdaArray \\
/MJ/eventaction/rootfilename testGerda.root \\
}
and start running.

That is it!

\bibliography{}
\begin{thebibliography}{3}
\bibitem{G4ref} S.~Agostinelli \emph{et al.}, \textsc{Geant4} -- a
simulation toolkit, Nucl. Instr. Meth. A, 506 (2003) 250
\bibitem{MJref} Majorana Monte Carlo User's and Developer's Guide, \\
http://neutrino.lbl.gov/majorana/udg/book1.html (internal)
\end{thebibliography}

\end{document}
