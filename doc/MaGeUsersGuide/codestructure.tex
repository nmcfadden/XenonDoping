\section{Introduction }

The simulation package is written in C++ and uses object-{}oriented
         programming techniques and the STL. 
         The geometry, tracking and physics package
         is Geant 4 based. A working knowledge of Geant 4 is required to 
         follow the discussions in this section. 
         Waveform simulation is performed with custom software.
 

\section{Program Structure}

       The program consists of sets of C++ classes that are grouped into
      software packages. All the software packages are built into a 
      single executable, \nolinkurl{MaGe}. The simulation is 
      setup by selecting detector geometries,
      physics processes, generators, etc. via text macros based on Geant 4 
      messengers. The software packages name's, with their corresponding CVS
      directory names in brackets, are given below: 
 

\begin{lstlisting}
  MaGe--
     |- Database (database)
     |- Generators (generators)
     |- Geometry (geometry)
     |- Gerda specific geometry (gerdageometry)
     |- Majorana specific geometry (mjgeometry)
     |- Output (io)
     |- Gerda specific output (gerdaio)
     |- Majorana specific output (mjio)
     |- Management (management)
     |- Materials (materials)
     |- Processes (processes)
     |- Waveform (waveform)
   \end{lstlisting}

      Note that not every CVS directory contains a software package, since 
     some CVS directories, ie. \nolinkurl{doc}, do not contain
     parts of the simulation program. Other packages, such as 
 \nolinkurl{Geometry}, are spread over several CVS directories
     to separate code from the Gerda and Majorana.  
     Each package is discussed in detail in subsequent chapters. We will give
     a brief summary of each package here, as well as how they are connected.
 

\noindent
\begin{description}
\item[{Database}] 

             Interfaces with the PostGreSQL database server.
            The database is used to store geometry data and calibration 
            constants and also contains all the materials definitions.
	    Currently only used by the Majorana experiment.
 


\item[{Generators}] 

             The different event generators. Currently several event
	    generators are supported and are user-{}selectable at runtime. See
 Chapter \ref{Physics}.
 


\item[{Geometry}] 

             The classes describing the different detector geometries. 
            It is possible to combine different geometries into a single 
            detector, ie. a NaI veto barrel around a clover detector, the
	    Gerda detector geoemetry, etc. 
            See Chapter \ref{Geometry}.
 


\item[{Output}] 

          The classes describing the different output for the 
         different detectors. Both AIDA and Root are supported. 
	 It is possible to have more than one output class
         per detector. 
	 Also contains the \nolinkurl{MGLogger} class 
         that handles terminal
         output and error messages. See Chapter \ref{IO}  


\item[{Management}] 

 	   Management classes, including 
 \nolinkurl{MaGe/management/MaGe.cc} that contains the 
 \texttt{m\-a\-i\-n\-(\-)}. See 
 Section \ref{Structure_Implement} for more information on the 
           management 
           classes.
 


\item[{Materials}] 

            These classes automatically build requested materials from elements
           that are in turn build from isotopes. All material, elemental and
           isotopic data are stored in the database. All new materials
           have to be added to the database. See Chapter \ref{Geometry}  


\item[{Processes}] 

            User-{}selectable Geant 4 physics lists for specific simulations, ie. 
           double-{}beta decay studies requires different physics process than
           dark matter studies. See Chapter \ref{Physics}.
 


\item[{Waveform}] 

            Waveform simulation package. Can be run independently from Geant 4. 
           Not yet implemented. See Chapter \ref{Waveforms}.
 



\end{description}

\section{How Everything is Implemented and 
Fits Together.}
\label{Structure_Implement}

       In the \nolinkurl{main()} function a collection
      of ``management" classes are instantiated. These classes control specific
      components of the framework. Classes can communicate with each other
      via the singleton \nolinkurl{MGManager} class. The 
 \nolinkurl{MGManager} class is instantiated 
      in the file \nolinkurl{MG/management/MaGe.cc}, and is 
      a collection of pointers to all the management classes.
      By using the \nolinkurl{MGManager} class, global variables are
      avoided.  
 

       The management class are listed below in the order they are instantiated:
 

\noindent
\begin{description}
\item[{\nolinkurl{MGGeometryDetectorConstruction}}] 

          Selects detector geometry via Geant 4 messenger and constructs it
         in Geant 4. Also constructs all required materials from the database,
	 if requested by the selected detector.
         See Chapter \ref{Geometry}. Accessed via:
 \texttt{M\-G\-M\-a\-n\-a\-g\-e\-r\-:\-:\-G\-e\-t\-M\-G\-G\-e\-o\-m\-e\-t\-r\-y\-D\-e\-t\-e\-c\-t\-o\-r\-C\-o\-n\-s\-t\-r\-u\-c\-t\-i\-o\-n\-(\-)}  


\item[{\nolinkurl{MGProcessesList}}] 

          Geant 4 physics list. User can select different physics packages
         via Geant 4 messengers. See Chapter \ref{Physics}. Accessed via:
 \texttt{M\-G\-M\-a\-n\-a\-g\-e\-r\-:\-:\-G\-e\-t\-M\-G\-P\-r\-o\-c\-e\-s\-s\-e\-s\-L\-i\-s\-t\-(\-)}  


\item[{\nolinkurl{MGManagementVisualization}}] 

          Registers Geant 4 visualization packages. Performs no other actions.
 


\item[{\nolinkurl{MGGeneratorPrimary}}] 

          Selects and manages event generators. See Chapter \ref{Physics}.
	 Accessed via:
 \texttt{M\-G\-M\-a\-n\-a\-g\-e\-r\-:\-:\-G\-e\-t\-M\-G\-G\-e\-n\-e\-r\-a\-t\-o\-r\-P\-r\-i\-m\-a\-r\-y\-(\-)}  


\item[{\nolinkurl{MGManagementRunAction}}] 

          Performs pre-{} and post run actions by 
         executing \texttt{B\-e\-g\-i\-n\-O\-f\-R\-u\-n\-A\-c\-t\-i\-o\-n\-(\-)} 	 and \texttt{E\-n\-d\-O\-f\-R\-u\-n\-A\-c\-t\-i\-o\-n\-(\-)} methods 
         of output
         classes. Derived from \nolinkurl{G4UserRunAction}.
	 Accessed via:
 \texttt{M\-G\-M\-a\-n\-a\-g\-e\-r\-:\-:\-G\-e\-t\-M\-G\-R\-u\-n\-A\-c\-t\-i\-o\-n\-(\-)}  


\item[{\nolinkurl{MGManagementEventAction}}] 

          Performs pre-{} and post event actions by 
         Executing \texttt{B\-e\-g\-i\-n\-O\-f\-E\-v\-e\-n\-t\-A\-c\-t\-i\-o\-n\-(\-)} 	 and \texttt{E\-n\-d\-O\-f\-E\-v\-e\-n\-t\-A\-c\-t\-i\-o\-n\-(\-)} methods 
         of output
         classes. Selects and contains pointer to Root output class 
         (Chapter \ref{IO}). 
         Derived from \nolinkurl{G4UserEventAction}.
	 Accessed via:
 \texttt{M\-G\-M\-a\-n\-a\-g\-e\-r\-:\-:\-G\-e\-t\-M\-G\-E\-v\-e\-n\-t\-A\-c\-t\-i\-o\-n\-(\-)}  


\item[{\nolinkurl{MGManagementSteppingAction}}] 

          Performs actions during each step of Geant 4 tracking by  
         executing \texttt{U\-s\-e\-r\-S\-t\-e\-p\-p\-i\-n\-g\-A\-c\-t\-i\-o\-n\-(\-)}          method of root output classes.
	 Derived from \nolinkurl{G4UserSteppingAction}.
	 Accessed via:
 \texttt{M\-G\-M\-a\-n\-a\-g\-e\-r\-:\-:\-G\-e\-t\-M\-G\-S\-t\-e\-p\-p\-i\-n\-g\-A\-c\-t\-i\-o\-n\-(\-)}  



\end{description}

       A pointer to the \texttt{M\-G\-M\-a\-n\-a\-g\-e\-r} singleton 
      class is accessible via the 
 \texttt{M\-G\-M\-a\-n\-a\-g\-e\-r\-:\-:\-G\-e\-t\-M\-G\-M\-a\-n\-a\-g\-e\-r\-(\-)} method.
      The Geant 4 visualization and run manager are also accessible via:
 \texttt{M\-G\-M\-a\-n\-a\-g\-e\-r\-:\-:\-G\-e\-t\-G\-4\-R\-u\-n\-M\-a\-n\-a\-g\-e\-r\-(\-)} and
 \texttt{M\-G\-M\-a\-n\-a\-g\-e\-r\-:\-:\-G\-e\-t\-G\-4\-V\-i\-s\-M\-a\-n\-a\-g\-e\-r\-(\-)} methods.
 

       There are two packages that are not accessible via the 
 \texttt{M\-G\-M\-a\-n\-a\-g\-e\-r}. The first is the Waveform
      package that is not implemented yet. It will be accessible via 
 \texttt{M\-G\-M\-a\-n\-a\-g\-e\-r} when it is implemented. 
      The other
      is the database. For historical reasons, it is accessed via it's own
      singleton control class: \texttt{M\-G\-D\-a\-t\-a\-b\-a\-s\-e}. See
 Chapter \ref{Database}.
 

\section{Programming Guidelines and Naming Conventions}

   MaGe requires the use of good object-{}oriented programming techniques.
  We strongly discourage the use of ``Fortranized" C++. Our coding conventions
  are based on modified  
 \href{http://root.cern.ch/root/Conventions.html}{Root} and 
 \href{http://pcroot.cern.ch/TaligentDocs/TaligentOnline/DocumentRoot/1.0/Docs/books/WM/WM_3.html}{   Taligent guides }. Enforcing too rigid 
  programming rules are not feasible, although we require the following 
  reasonable conventions:
 
\begin{itemize}

\item{}           All class names begin with \texttt{G\-E\-, M\-G},
	  or \texttt{M\-G}. 
 \texttt{G\-E} is reserved for classes specific
	  to the Gerda simulation, such as geometries. 
 \texttt{M\-J} is specfically for Majorana 
	  classes. Common classes, such as management, start with 
 \texttt{M\-G}. If in doubt, use 
 \texttt{M\-G}.
 


\item{}           All virtual base class names begin with 
 \texttt{M\-G\-V\-, M\-G\-V\-, G\-E\-V}.
 


\item{}           All class member variables names begin with 
 \texttt{f}  


\item{}           All local automatic variables names must begin with a lowercase, ie 
 \texttt{s\-o\-m\-e\-L\-o\-c\-a\-l\-R\-a\-n\-d\-o\-m\-N\-u\-m\-b\-e\-r}  


\item{}          All variable, class names, etc. should have descriptive, 
         written out names. Think carefully about naming and avoid ambiguity.
 


\item{}          All unitialized pointers should be set to 0 or 
 \texttt{N\-U\-L\-L}.
 


\item{}          All terminal output should be done via the 
 \nolinkurl{MGLogger} class, if possible. 
	 See Section \ref{IO_MGLogger}. 
 


\item{}          Code should be indented 3 spaces per nesting level. 
 


\item{}          Lines longer than 80 columns should be wrapped with a carriage return.
 


\item{}          The use of a separator between methods, ie.
 

\begin{lstlisting}
//---------------------------------------------------------------------------//
\end{lstlisting}

          is strongly encouraged.
 

\end{itemize}

    Some of the older code in the \nolinkurl{MaGe}    package do not follow these rules, since they were written before the rules 
   were made explicit. Please do not follow their conventions. If you 
   encounter any cases not covered here, use the Root/Taligent guideline, 
   use you best judgment, or contact Reyco Henning (Majorana) or 
   Xiang Liu (Gerda).
 
%\begin{DBKadmonition}{warning}{Important}
% FixME there should be some sort of admonition added here
\emph{Remember that you are not coding for yourself! Many others will have to read you code. Write the code the way you would like to read it!}  
%\end{DBKadmonition}


