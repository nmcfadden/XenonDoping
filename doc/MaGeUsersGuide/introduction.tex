This note is addressed to new users of the \gerda\ Monte Carlo
software \mage\ and those users who would like to become software
developers. It is written with the intention to (1) teach non-experts
how to install and run \mage\ and, for developers, to (2) introduce
the guidelines and structure of the C++ code. For non-experts, only a
basic knowledge of \rootv and C++ is needed in order to interpret the
data created with the Monte Carlo package. Developers are assumed to
be familiar with C++ and object-oriented programming. \\ 

The \mage\ software is developed and used by \majorana\ and \gerda.
This note is focused on the \gerda\ parts of the software and the
parts common to both experiments. \\

The simulation of physics processes is motivated by several needs. The
most important one addressed before the actual start of the experiment
are an estimate on the experimental situation to be encountered. For
\gerda\ this includes e.g. the evaluation of the background index and
the calculation of the sensitivity. In addition, the simulation can
guide the design of detector components. After data becomes available
the simulation of the experimental setup has to be verified and can,
in a second step, be used to interpret these data. This holds true for
\gerda\ and for the test stands associated with it. It is these three
requirements which led to development of the software package
\mage. \\

The choice of the framework within which the \gerda\ Monte Carlo
simulation is realized is \geant~\cite{Agostinelli:2002hh}. It is a
well established simulation code with a large user community and a
broad range of physics applications~\cite{Allison:2006ve}. The
software features the full chain of simulation, from the generation of
particles to tracking and read-out. The physics processes implemented
in \geant\ are validated and/or
parameterized~\cite{geant4physics,Amako:2005xf,Amako:2006nb}. As it
provides full flexibility due to object-oriented C++-programming it
meets the technical requirements for the realization of the needs
described earlier. \\

The \mage\ software is based on \geant\ libraries and was initially
written by the \majorana\ Monte Carlo group. It is now developed and
maintained by both, the \gerda\ and \majorana\ Monte Carlo groups. As
both collaborations aim at a measurement of neutrinoless double
beta-decay of $^{76}$Ge, the physics processes encountered in the
experiments are the same. With a common Monte Carlo software a
duplication of work is minimized.The maintainance of the code through 
more developers and the physics validation is made easier as more
data is and will be available. On the other hand, the \mage\ software
is flexible enough to provide both collaborations with their
particular needs like the geometries, physics processes, etc. \\

\mage\ features a non-expert mode of running via macros maintaining a
maximum of flexibility. Geometries, physics processes or the output
scheme are individually adjustable without recompiling the code. On the
other hand, due to the object-oriented structure, it is easy for
developers to add patches such a new test stand geometries, additional
physics lists, etc. \\

Part~I of this note is addressed to the non-experts. The installation of
\mage\ (chapter~\ref{chapter:installation}) and the documentation
system (chapter~\ref{chapter:documentation}) are described. Also, the
running of simple macros to simulate events or to display the selected
geometry are introduced (chapter~\ref{chapter:macros}). The generation
of events which are needed as input parameters for the simulation is
discussed in chapter~\ref{chapter:generators}. \\

Part~II is is addressed to \mage\ developers. It starts with the basics
common to \majorana\ and \gerda\ such as compiling the code
(chapter~\ref{chapter:compiling}) and the implemented physics lists
(chapter~\ref{chapter:physics}). This is followed by the \gerda\
specific parts like the description of the different I/O schemes
available (chapter~\ref{chapter:io}) and the \gerda\ detector
components (chapter~\ref{chapter:detector}). The implementation of
test stand environments is explained in
chapter~\ref{chapter:teststands}. \\

As mentioned earlier, \mage\ is developed by people in Europe and the
U.S. It is therefore important to obey some defined rules when it
comes to coding. The three most important ones are
%
\begin{itemize} 
\item Each developer is responsible for his or her code. If questions 
arise or bugs are reported it is their duty to maintain and, if
necessary, modify the code.
% 
\item Each developer is asked to document his or her changes made to 
the code. This includes comments in the code itself (header and
in-line) and a description in the documentation system. 
% 
\item There are three parts which are separated and marked with prefixes. 
These are the common part (MG), the \majorana\ part (MJ) and the
\gerda\ part (GE). It is understood that only the experiment specific
(here: \gerda) code is to be edited independently. For changes to the
common part both Monte Carlo groups have to be informed. 
\end{itemize} 
